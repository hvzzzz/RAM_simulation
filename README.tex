% Created 2022-07-17 Tun 12:16
% Intended LaTeX compiler: pdflatex
\documentclass[11pt]{article}
\usepackage[utf8]{inputenc}
\usepackage[T1]{fontenc}
\usepackage{graphicx}
\usepackage{longtable}
\usepackage{wrapfig}
\usepackage{rotating}
\usepackage[normalem]{ulem}
\usepackage{amsmath}
\usepackage{amssymb}
\usepackage{capt-of}
\usepackage{hyperref}
\author{Hanan Quispe}
\date{2022}
\title{RAM(Robot de Afecciones Múltiples)-Simulation}
\hypersetup{
 pdfauthor={Hanan Quispe},
 pdftitle={RAM(Robot de Afecciones Múltiples)-Simulation},
 pdfkeywords={},
 pdfsubject={},
 pdfcreator={Emacs 27.2.50 (Org mode 9.5.4)}, 
 pdflang={English}}
\begin{document}

\maketitle
Este repositorio contiene los scripts para ejecutar el demo del navigation stack usando Turtlebot3 y ROS(Robot Operating System), además de ello también contiene una interfaz gráfica para el control del robot desde una red local.

\section{Dependencias}
\label{sec:org683f4b4}
\subsection{ROS}
\label{sec:orgad24d42}
La versión de ros utilizada es ROS Melodic esta se puede instalar siguiendo las instrucciones del siguiente enlace \href{http://wiki.ros.org/melodic/Installation/Ubuntu}{ROS Melodic}.
\subsection{Create React App}
\label{sec:orgf81c8e5}
Referirse a \href{https://github.com/facebook/create-react-app}{Create React App} para instalar las dependencias correspondientes.

\subsection{Dependencias Adicionales}
\label{sec:orga8c49e3}
Estas se instalan usando 
\begin{verbatim}
bash install.sh
\end{verbatim}
\section{Ejecución del Simulador}
\label{sec:org6945fc5}
La simulación se ejecuta en dos computadoras, en una de ellas se ejecuta ROS con la simulacion de Turtlebot3 y en la segunda se ejecuta la interfaz de control.
\end{document}